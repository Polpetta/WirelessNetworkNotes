\chapter{Molecular Communication}

Molecular communication is performed through nano machines: a nano-scale device
able to perform a specific task at nano-level.
It can be a natural device or an artificial one. Each one has specific
characteristics, but both work in a similar way.

The development of this devices can have different approaches:
\begin{itemize}
\item Top-down $\to$ down-scaling existing components
\item Bottom-up $\to$ develop using individual molecules (they grow developing
  them self)
\item Bio-hybrid
\end{itemize}

Molecular communications found applications in different environments, such as:
\begin{itemize}
\item Oil spilling containment
\item Health monitoring
\item etc...
\end{itemize}
All of these applications should be highly reliable and secure.

\section{Nano-network}

A nano network is composed by a set of components on a nano-scale and their
interconnection. Usually, these nano networks are more biological than
electronic.
There are different ways communications are carried on:
\begin{itemize}
\item Standard communication
\item Nano-mechanical communication
\item Molecular communication
\end{itemize}

\subsection{Standard communication}

There are two types of standard communications:
\begin{enumerate}
\item Electromagnetic waves
  \begin{itemize}
  \item wiring a large quantity of nano-machines is unfeasible
  \item wireless solution could be used but antennas are hard to be integrated
  \end{itemize}
\item Acoustic waves
\end{enumerate}

\subsection{Nano-mechanical communication}

With this kind of communication you need to have junctions between the
nano-machines, to perform actions like DNA transfer. Having physical contacts
between two nano-machines isn't always easy.

\subsection{Molecular communication}

Molecular communication is the more feasible solution for the knowledge humanity
has today.
The information and the transmission are encoded in molecules, and it's the
most promising approach for nano networking.

There are two types of transmission: \textbf{short range} (within mm)
\textbf{long range} (up to km).

\subsubsection{Short range transmission}

Short range transmission can happen through molecular motors or calcium
signalling. With molecular motors proteins can transform chemical energy into
mechanical work. They can travel along molecular rails, and data need to be
encoded, transmitted and decoded somehow. On the other hand, calcium is a
well-known molecular for communication, as it's responsible for many coordinated
cellular tasks (like messages that triggers an action). This is more flexible
than railways communication, and different kind of signals can be specified
using different levels of calcium.

\subsubsection{Long range transmission}

Long range communications can use pheromones or bacteria: in the first case the
pheromones are molecular compounds containing information that can only be
decoded by specific receivers, and messages consist of molecules.
Bacteria have interesting characteristics: conjugation (bacteria interconnection
that allows plasmoids transmission), chemotaxis (movement induced into bacteria
by chemical stimuli), and can develop antibiotics resistance, that can be useful
to get rid of interference when there are too many bacteria.
